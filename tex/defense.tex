\documentclass[aspectratio=169]{beamer}

% \setbeameroption{show only notes}
% \setbeameroption{show notes}
\usetheme{PaperPlane}
% \usetheme{Darmstadt}


% Ensure one notes slide per content slide
\makeatletter
\def\beamer@framenotesbegin{
	\gdef\beamer@noteitems{}%
	\gdef\beamer@notes{{}}%
}
\makeatother

% Don't number the title slide
\let\otp\titlepage
\renewcommand{\titlepage}{\otp\addtocounter{framenumber}{-1}}

% Math operators
% \DeclareMathOperator*{\argmax}{arg\,max}
% \DeclareMathOperator*{\argmin}{arg\,min}
% \DeclareMathOperator{\permindex}{index}
% \DeclareMathOperator{\Inv}{Inv}
\DeclareMathOperator{\probability}{P}
% \DeclareMathOperator{\expectedvalue}{E}
% \DeclareMathOperator{\meet}{\wedge}
% \DeclareMathOperator{\bigmeet}{\bigwedge}
% \DeclareMathOperator{\join}{\vee}
% \DeclareMathOperator{\bigjoin}{\bigvee}
% \DeclareMathOperator{\symmetricdifference}{\vartriangle}


\title{A Generalized Probabilistic Gibbard-Satterthwaite Theorem}
\author{Jonathan Potter}
\institute[RIT]{Rochester Institute of Technology}

\begin{document}
	\begin{frame}[plain]
		\maketitle
	\end{frame}

	\section{Backround}

		\begin{frame}
			\frametitle{Importance of Elections}
			\begin{itemize}
				\item Political voting
				\item Electing board members
				\item Shareholders voting on company issues
				\item Artificial intelligent agent decision
				\item Search engine page-ranking
			\end{itemize}

			\note{Why are elections important? They can be applied any time independent agents need to come to a consensus.}
		\end{frame}

		\begin{frame}
			\frametitle{Desirable Election Systems}
			\begin{itemize}
				\item Fairness
				\begin{itemize}
					\item Everyone should have an equal say
					\item Winner should accurately represent the group
					\item Simple with 2 alternatives; complicated with many
					\item First studied by Condorcet
				\end{itemize}
			\end{itemize}

			\note{
				With two alternatives the one prefered by the most voters should win.

				He proposed that the winning candidate be the candidate who would win a head-to-head election against each of the other candidates.

				Such a winner is known as the Condorcet winner.

				Unfortunately, Condorcet also proved that a Condorcet winner does not always exist.

				But this was one of the first fairness criterion.
			}
		\end{frame}

		\begin{frame}
			\frametitle{Arrow's Impossibility Theorem}
			\begin{itemize}
				\item Unrestricted domain (universality)
				\item Independence of irrelevant alternatives
				\item Pareto principle (unanimity)
				\item Non-dictatorship
			\end{itemize}

			\note{
				No social welfare function (election system) with 3 or more alternatives can satisfy all of these.

				These all seem like obvious basic requirements for a ``fair'' election system.

				Unrestricted domain: All individual preferences are allowed and yield a valid group preference.

				Independence of irrelevant alternatives: If all voters' preferences between alternatives x and y remain the same, the group preference between x and y is unchanged even if voters change their preferences regarding other alternatives.

				Pareto principle (unanimity): Unanimity of individual preferences implies a group preference. E.g. if all individuals prefer alternative x to y, then the group will prefer x to y.

				Non-dictatorship: There is no voter whose preference always dictates the group's preference.

				So we have to compromise.
			}
		\end{frame}

		\begin{frame}
			\frametitle{Manipulation}
			A voter can get a better result by voting strategically, rather than voting his actual preferences.

			\vspace{2em}
			\hspace*{1.2cm}{\Large Real Preferences}\hspace{5cm}{\Large Manipulation}

			\vspace{.5em}
			\centerline{\includegraphics[height=0.4\paperheight, keepaspectratio]{../figures/manipulation_example.pdf}}

			\note{
				Assume a simple plurality/first-past-the-post system.

				In the first case it would be a tie, but lets assume our arbitrary tie breaking technique will choose the Democrats.

				This is essentially the ``wasted vote'' problem.

				This is a very simplistic example, but almost all voting systems are susceptible to manipulation.
			}
		\end{frame}

		\begin{frame}
			\frametitle{Gibbard-Satterthwaite Theorem}
			Voting rules are manipulable if they satisfy:
			\begin{itemize}
				\item \textbf{Non-dictatorship}: No single voter always dictates the group preference.
				\item \textbf{Non-imposition}: Every alternative has the possibility of winning.
			\end{itemize}

			\note{
				They must satisfy both of these.
			}
		\end{frame}


	\section{Preliminaries}
	\subsection{Notation}

		\begin{frame}
			\frametitle{Notation}
			\begin{itemize}
				\item $C = \{1, \ldots, m\}$ is the set of alternatives.
				\item A preference list is a total ordering of the alternatives.
				\item The set of all preference lists is $L(C)$.
				\item A preference profile is a sequence of $n$ preference lists.
				\item The set of all preference profiles is $P = L(C)^n$.
				\item A voting rule is a function that chooses a winning alternative from a profile, i.e. $f : P \to C$.
				\item An election is a voting rule paired with a profile: $(f, p)$.
			\end{itemize}

			\note{
				$m$ is the number of alternatives

				$n$ is the number of voters

				Voting rule == social choice function

				$L(C)^n$ is the Cartesian product of $L(C)$ with itself $n$ times, or the set of all $n$-tuples of elements of $L(C)$
			}
		\end{frame}

		\begin{frame}
			\frametitle{Restricted Preference Profiles}
			\begin{itemize}
				\item For a preference list $v$, $v|_D$ means $v$ \emph{restricted to} $D$, i.e. $v$ with the alternatives from $D$ removed.
				\item For a preference profile, $p$, $p|_D$ means $p$ with each preference list restricted to $D$.
			\end{itemize}

			\hspace*{3.5cm}{\Large $p$}\hspace{6.2cm}{\Large $p|_{\{1, 2\}}$}

			\vspace{.5em}
			\centerline{\includegraphics[height=0.3\paperheight, keepaspectratio]{../figures/restriction_example.pdf}}
		\end{frame}


	\subsection{Proof Summary}

		\begin{frame}
			\frametitle{Proof Summary}
			Friedgut's proof is in three steps:
			\begin{description}
				\item[Step 1:] application of a quantitative version of Arrow's impossibility theorem.
				\item[Step 2:] reduction from an SCF with low dependence on irrelevant alternatives to a GSWF with a low paradox probability.
				\item[Step 3:] reduction from low manipulation power to low dependence on irrelevant alternatives.
			\end{description}
		\end{frame}

		\begin{frame}
			\frametitle{Generalized Steps}

			Friedgut, was able to generalize Step 1 and Step 2 as follows:

			\begin{lemma}[Generalized Step 1]
				For every fixed $m$ and $\epsilon > 0$ there exists $\delta > 0$ such that if $F = f^{\otimes \binom{m}{2}}$ is a neutral IIA GSWF over $m$ alternatives with $f : \{0,1\}^n \rightarrow \{0,1\}$, and $\Delta(f, DICT) > \epsilon$, then $F$ has probability of at least $\delta \ge (C\epsilon)^{\lfloor m/3 \rfloor}$ of not having a Generalized Condorcet Winner, where $C > 0$ is an absolute constant.
			\end{lemma}

			\begin{lemma}[Generalized Step 2]
				For every fixed $m$ there exists $\delta > 0$ such that for all $\epsilon > 0$ the following holds. Let $f$ be a neutral SCF among $m$ alternatives such that $\Delta(f, DICT) > \epsilon$. Then for all $(a,b)$ we have $M^{a,b}(f) \ge \delta$.
			\end{lemma}

			\note{
				So we only need to generalize step 3 to get the whole proof.
			}
		\end{frame}

		\begin{frame}
			\frametitle{Step 3}

			The original Step 3 was:

			\begin{lemma}[Non-General Step 3]
				For every SCF $f$ on $3$ alternatives and every $a,b \in A$, $M^{a,b} \le \sum_i M_i \cdot 6$
			\end{lemma}

			And the generalization we attempt to prove is:

			\begin{lemma}[Generalized Step 3]
				For every SCF $f$ on $m$ alternatives and every $a,b \in A$, $M^{a,b} \le \sum_i M_i \cdot m!$
			\end{lemma}
		\end{frame}

		\begin{frame}
			\frametitle{Main Result}

			When we put together all 3 generalized steps we get our main result:

			\begin{theorem}[Main Result]
				There exists a constant $C > 0$ such that for every $\epsilon > 0$ the following holds. If $f$ is a neutral SCF for $n$ voters over $m$ alternatives and $\Delta(f, g) > \epsilon$ for any dictatorship $g$, then $f$ has total manipulability: $\sum^n_{i=1} M_i(f) \ge \frac{(C\epsilon)^{\lfloor m/3 \rfloor}}{m!}$.
			\end{theorem}

			Step 3 is comprised of Lemma 6, Lemma 7, and Lemma 8 which we will generalize one at a time.
		\end{frame}

	\section{Results}
	\subsection{Lemma 6}

		\begin{frame}
			\frametitle{Statement of Lemma 6}

			\begin{lemma}[Original Lemma 6]
				\[
					M^{a,b}(f) = E_{q \in L(C)^n} \left[ \frac{|A(q)|}{3^n} \cdot \frac{|B(q)|}{3^n} \right],
				\]
				where $q$ is chosen uniformly at random.
			\end{lemma}

			\begin{lemma}[Generalized Lemma 6]
				\[
					M^{a,b}(f) = E_{q \in L(C)^n} \left[ \frac{|A(q)|}{\left(\frac{m!}{2}\right)^n} \cdot \frac{|B(q)|}{\left(\frac{m!}{2}\right)^n} \right],
				\]
				where $q$ is chosen uniformly at random.
			\end{lemma}
		\end{frame}

		\begin{frame}
			\frametitle{Define $A$ and $B$ Functions}

			Let $a, b \in C$ be the first two alternatives, let $p \in L(C)^n$ be a preference profile. We define
			\begin{align*}
				A(p) &= \{x \in L(C)^n \mid x|_{\{a,b\}} = p|_{\{a,b\}}, f(x) = a\} \\
				B(p) &= \{x \in L(C)^n \mid x|_{\{a,b\}} = p|_{\{a,b\}}, f(x) = b\}.
			\end{align*}
		\end{frame}

		\begin{frame}
			\frametitle{Define $M^{a,b}$}

			Recall the definition of $M^{a,b}(f)$ from Friedgut:
			\[
				M^{a,b}(f) = \probability(f(p) = a, f(p') = b)
			\]
			where $p, p'$ are chosen at random in $L(C)^n$ with $p|_{\{a,b\}} = p'|_{\{a,b\}}$.
		\end{frame}

		\begin{frame}
			\frametitle{Size of Profiles}

			For any preference profile $p \in P$ there are $(\frac{m!}{2})^n$ profiles $x$ such that $x|_{\{a, b\}} = p|_{\{a, b\}}$. This is because there are $m!$ possible preference lists; half of them will have the preference between $a$ and $b$ that agrees with $p|_{\{a, b\}}$ and half will disagree. This gives $\frac{m!}{2}$ possible preference lists for each voter, so there are $(\frac{m!}{2})^n$ profiles comprised of these preference lists.
		\end{frame}

		\begin{frame}
			\frametitle{Proof of Generalized Lemma 6}

			First we fix a profile $q$. Then
			\[
				\frac{|A(q)|}{\left(\frac{m!}{2}\right)^n}
			\]
			is the probability that a randomly chosen profile, $p$, satisfying $p|_{\{a,b\}} = q|_{\{a,b\}}$ also satisfies $f(p) = a$. This is because there are $(\frac{m!}{2})^n$ profiles that agree with $q|_{\{a,b\}}$, and $|A(q)|$ is the number of those for which the outcome is $a$.

			Since $p|_{\{a,b\}} = q|_{\{a,b\}}$ and $p'|_{\{a,b\}} = q|_{\{a,b\}}$, clearly we have that $p|_{\{a,b\}} = p'|_{\{a,b\}}$. Since $f(p) = a$ and $f(p') = b$ are independent events, the joint probability is
			\[
				\frac{|A(q)|}{\left(\frac{m!}{2}\right)^n} \cdot \frac{|B(q)|}{\left(\frac{m!}{2}\right)^n}.
			\]
		\end{frame}

		\begin{frame}
			\frametitle{Proof of Generalized Lemma 6}

			So we can rewrite
			\[
				M^{a,b}(f) = \probability(f(p) = a, f(p') = b)
			\]
			as
			\[
				M^{a,b}(f) = E_{q \in L(C)^n} \left[ \frac{|A(q)|}{\left(\frac{m!}{2}\right)^n} \cdot \frac{|B(q)|}{\left(\frac{m!}{2}\right)^n} \right],
			\]
			\qed
		\end{frame}

		\begin{frame}
			\frametitle{Independent Work}
			\begin{itemize}
				\item Placeholder 
			\end{itemize}
		\end{frame}

		\begin{frame}
			\frametitle{Questions}
			\begin{itemize}
				\item Placeholder
			\end{itemize}
		\end{frame}
\end{document}
